\nochaptercolor{mygrey}{Nomenclatures}
    \section*{\textcolor{mygrey}{Nomenclatures}}
		{\noindent}Ceci est une liste des nomenclatures utilisées dans ces pages.
		\begin{description}
		    \item[Gènes]     										\hfill \\
		        La nomenclature \acs{HGNC} est utilisée pour le nom des gènes. Le symbole officiel du gène est utilisé tout au long de ce document. Pour écrire le symbole du gène nous utilisons la nomenclature usuelle avec le symbole du gène en majuscule et en italique (exemple : \acs{TP53}). Le nom complet officiel est détaillé dans les abréviations \ref{app:ac:gènes}.
		    \item[Protéines] 										\hfill \\
		        Le symbole officiel du gène dont est issue la protéine est utilisé tout au long de ce document. Pour écrire le symbole de la protéine, nous utilisons la nomenclature usuelle avec le symbole du gène en majuscule sans italique (exemple : \acs{p-TP53}). Le nom complet recommandé par le consortium UniProt est détaillé dans les abréviations \ref{app:ac:protéines}.
		\end{description}