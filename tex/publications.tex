\nochaptercolor{white!15!black}{Publications}
  \section*{\textcolor{white!15!black}{Chapitres}}
    \subsection*{\textcolor{white!15!black}{Garcia et al. 2011}}\label{app:Garcia2011}
      Dans ce chapitre \citep{Garcia2011}, nous décrivons en détail l'algorithme \acs{ITIfr} ainsi que la façon dont on l'utilise pour réaliser une analyse non-supervisée.

      \begin{description}
        \item [Abstract]                            \hfill \\
          \url{www.igi-global.com/chapter/linking-interactome-disease/52327}
        \item [Base de données des sous-réseaux]    \hfill \\
          \url{http://iti.sourceforge.net/unsupervised-10-datasets/index.html}
        \item [Documentation]                       \hfill \\
          \url{http://sourceforge.net/p/iti/wiki/Home/}
        \item [Code Source]                         \hfill \\
          \url{http://sourceforge.net/projects/iti/files/Source%20Code/iti-1.0.tar.gz}
      \end{description}

      \includepdf[pages={3}]{articles/Garcia2011.pdf}
      %\includepdf[pages={3-24}]{articles/Garcia2011.pdf}

      \subsection*{\textcolor{white!15!black}{Garcia et al. soumis}}\label{app:Garcia2013}
        Dans ce chapitre \citep{Garcia2013}, nous détaillons point par point la façon dont on utilise \acs{ITIfr} pour réaliser une analyse supervisée.

        \begin{description}
          \item [Base de données des sous-réseaux]    \hfill \\
            \url{http://iti.sourceforge.net/supervised-5-datasets/index.html}
          \item [Documentation]                       \hfill \\
            \url{http://sourceforge.net/p/iti/wiki/Home/}
          \item [Code Source]                         \hfill \\
            \url{http://sourceforge.net/projects/iti/files/Source%20Code/iti-2.0.tar.gz}
        \end{description}

        \includepdf[pages={1}]{articles/Garcia2013.pdf}
        %\includepdf[pages={1-26}]{articles/Garcia2013.pdf}

      \subsection*{\textcolor{white!15!black}{Garcia et al. soumis}}\label{app:Garcia2013b}
        Dans ce chapitre \citep{Garcia2013b}, nous explorons l'intégration supplémentaires des CGH.

        \begin{description}
          \item [Base de données des sous-réseaux]    \hfill \\
            \url{http://iti.sourceforge.net/supervised-5-datasets/index.html}
          \item [Documentation]                       \hfill \\
            \url{http://sourceforge.net/p/iti/wiki/Home/}
          \item [Code Source]                         \hfill \\
            \url{http://sourceforge.net/projects/iti/files/Source%20Code/iti-2.0.tar.gz}
        \end{description}

        \includepdf[pages={1}]{articles/Garcia2013b.pdf}
        %\includepdf[pages={1-34}]{articles/Garcia2013b.pdf}

  \section*{\textcolor{white!15!black}{Article}}
    \subsection*{\textcolor{white!15!black}{Garcia et al. 2012}}\label{app:Garcia2012}
      Dans cet article \citep{Garcia2012}, nous décrivons en détail l'algorithme \acs{ITIfr}, ainsi que la façon dont on l'utilise pour réaliser une analyse supervisée.

        \begin{description}
          \item [Abstract]    \hfill \\
            \url{http://bioinformatics.oxfordjournals.org/content/28/5/672.abstract}
          \item [Base de données des sous-réseaux]    \hfill \\
            \url{http://iti.sourceforge.net/supervised-5-datasets/index.html}
          \item [Documentation]                       \hfill \\
            \url{http://sourceforge.net/p/iti/wiki/Home/}
          \item [Code Source]                         \hfill \\
            \url{http://sourceforge.net/projects/iti/files/Source%20Code/iti-2.0.tar.gz}
          \item [Companion web-site]    \hfill \\
            \url{http://iti.sourceforge.net}
        \end{description}

    \includepdf[pages={1}]{articles/Garcia2012.pdf}
    %\includepdf[pages={1-7}]{articles/Garcia2012.pdf}